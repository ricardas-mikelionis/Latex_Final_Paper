\documentclass{VUMIFPSbakalaurinis}
\usepackage{algorithmicx}
\usepackage{algorithm}
\usepackage{algpseudocode}
\usepackage{amsfonts}
\usepackage{amsmath}
\usepackage{bm}
\usepackage{caption}
\usepackage{color}
\usepackage{float}
\usepackage{graphicx}
\usepackage{listings}
\usepackage{subfig}
\usepackage{wrapfig}

% Titulinio aprašas
\university{Vilniaus universitetas}
\faculty{Matematikos ir informatikos fakultetas}
\department{Programų sistemų katedra}
\papertype{Bakalauro darbas}
\title{Neuroninių tinklų panaudojimas testavimo proceso gerinimui}
\titleineng{Using neural networks to improve the testing process}
\author{Ričardas Mikelionis}
% \secondauthor{Vardonis Pavardonis}   % Pridėti antrą autorių
\supervisor{asist. dr. Vytautas Valaitis}
\reviewer{partn. prof., dr. Aldas Glemža}
\date{Vilnius – \the\year}

% Nustatymai
\setmainfont{Palemonas}   % Pakeisti teksto šriftą į Palemonas (turi būti įdiegtas sistemoje)
\bibliography{bibliografija}

\begin{document}
\maketitle

%% Padėkų skyrius
% \sectionnonumnocontent{}
% \vspace{7cm}
% \begin{center}
%     Padėkos asmenims ir/ar organizacijoms
% \end{center}

\sectionnonumnocontent{Santrauka}
Glaustai aprašomas darbo turinys: pristatoma nagrinėta problema ir padarytos
išvados. Santraukos apimtis ne didesnė nei 0,5 puslapio. Santraukų gale
nurodomi darbo raktiniai žodžiai. 
% Nurodomi iki 5 svarbiausių temos raktinių žodžių (terminų).
% Vienas terminas gali susidėti iš kelių žodžių.
\raktiniaizodziai{raktinis žodis 1, raktinis žodis 2, raktinis žodis 3, raktinis žodis 4, raktinis žodis 5}   

\sectionnonumnocontent{Summary}
Santrauka anglų kalba. Santraukos apimtis ne didesnė nei 0,5 puslapio.
\keywords{keyword 1, keyword 2, keyword 3, keyword 4, keyword 5}

\tableofcontents

\sectionnonum{Įvadas}
Smarkiai augant Continuous Delivery principų populiarumui auga poreikis testuoti daugiau per trumpesnį laiko tarpą. Paprastas sprendimas šiai problemai būtų be abejo nuolat besisukantis automatinių testų paketas, tačiau automatiniais testais padengti programinį kodą 100\% praktiškai neįmanoma. 100\% padengimas įmanomas tik pagal kokią nors specifinę metriką, o vaikytis tikrojo 100\% padengtumo testais pagal kiekvieną metriką ar funkcinę sritį prilygsta šviesos geičio vaikymuisi, kuo arčiau esame tikslo tuo daugiau pastangų ir resursų reikia pasistūmėti į priekį. Todėl, žinoma, vis dar išlieka poreikis rankiniam testavimui. Norint išleisti programinės įrangos versiją kuo dažniau ištestuoti visko kiekvieną kartą neįmanoma. Taip šis problemos sprendimas iškelia dar vieną, mažesnę, problemą: kaip efektyviai pasirinkti testavimo sritis kiekvienai naujai programos laidai (angl. release). Alan M. Davis \cite{Davis:1995:PSD:203406} tegia, jog pareto principą galima pritaikyti ir programinės įrangos testavime: čia 80\% kode esančių defektų surandami 20\% viso kodo.

Tobulame pasaulyje iteracijai pasirinktos testavimo sritys ir apims tuos 20\% problematiškojo kodo. Tačiau dažnai net remiantis visa turima istorine informacija testuotojui atsakingam už testavimo sričių parinkimą yra sunku efektyviai atrinkti testavimo sritis, kuriose atliktas darbas turės didžiausią įtaką programinės įrangos kokybei.

Sieniant sėkmingai įgyvendinti darbo tikslą siekiama įgyvendinti šiuos uždavinius:
\begin{enumerate}
\item Išrinkti metrikas darančias įtaką programinės įrangos sudėtingumui
\item Palyginti efektyvumą tarp sprendimų medžio besimokančios mašinos ir neuroninio tinklo naudojant surinktus duomenis
\item Atrinkti duomenis turinčius mažiausią įtaką galutiniam spėjimui
\item Naudojantis mažesniu atributų kiekiu iš naujo apmokyti neuroninį tinklą ir palyginti rezultatus su pilnų duomenų rinkinių apmokyto neuroninio tinklo
\end{enumerate}
 
\section{Programinio kodo sudėtingumo ir defektyvumo sąsajos}
Kiekvieną kartą renkatis testuojamas programinės įrangos vietas, žinoma, jei nėra testuojamas visas programinės įrangos funkcionalumas, reikia turėti atskaitos tašką, kuris leistų nuspręsti kurios programiniės įrangos kodo vietos, ar programų paketo moduliai turi didžiausią riziką būti defektyvūs. Į šią kategoriją dažnai pakliūva seniai testuoti, nauji, atnaujinti bei anksčiau daug defektų turėję programinės įrangos moduliai. Didelė problema, su kuria vis dažniau susiduriama yra ką daryti, jei nėra laiko ištestuoti visiems šiems moduliams ir kaip tuomet pasirinkti atitinkamus progaminės įrangos modulius testavimui, kad nešvaistydami laiko padarytumėme didžiausią įmanomą įtaką programinės įrangos kokybės užtikrinimui. Reikėtų sukurti detalesnę programinės įrangos metrikų sistemą nei, „naujumas“ ar „prieš tai buvusių defektų kiekis“, kuri tiesiogiai atspindėtų būsimą (ar esamą) programinės įrangos defektyvumą. O gal būt užtenka pritaikyti egzistuojančias programinio kodo metrikas randant koreliaciją su programų defektyvumu.

Ir iš tiesų, būtų galima teigti, jog programinės kokybės metrikos kurias būtų įmanojma tiesiogiai susieti su programinio kodo defektyvumu egzituoja. Arčiausiai tokių metrikų yra programino kodo sudėtingumo metrikos. Be abejo yra ne vienas būdas matuoti programos kodo sudėtingumą, tačiau, šiam tyrimui pasirinktos dvi, galima būtų sakyti, populiariausios metrikos kodo sudėtingumui atvaizduotii: Maurice H. Halstead aprašytosios 1977-aisiais \cite{Halstead:1977:ESS:540137} bei Thomas J. McCabe aprašytosios 1976-aisiais \cite{McCabe:1976:CM:800253.807712}. 

Ryšys tarp kodo sudėtingumo bei defektų buvimo jame, o gal būt derėtų sakyti defektų buvimo galimybės kode, atrodytų, gana logiškas: kuo sudėtingesnis kodas, tuo sunkiau jį skaityti, kuo sunkiau kodas skaitomas tuo sunkiau jį plėsti, todėl kyla rizika defektų atsiradimui. Tokia prielaida tiriama nuo pat sudėtingumo metrikų aprašymo. Žinoma, dėl kodo sudėtingumo tiksliai defektų kiekui daromo efekto kyla nestuarimų. Atsiranda teigiančių, jog Halstead metrikos neturi jokio įrodomo ryšio su defektų buvimu, tačiau yra ir tyrimų pagrindžiančių kodo sudėtingumo bei defektų buvimo jame koreliaciją \cite{Schroeder1999APG}. Dažniausiai tyrėjai lieka šio argumento viduryje nei visiškai neigdami, nei patvirtindami šią koreliaciją. Gana dažnas sprendimas yra pamažinti metrikų kiekį ir patikrinti metrikas priklausomai nuo situacijos, taip užtikrinant, kad pasirinktosios metrikos tikrai korelioja su tyrimo metu ieškomu fenomenu, šiuo atveju kodo defektyvumu \cite{Metrics in Evaluating Software Defects:2013}.

Šiame dokumente aprašomame tyrime bus naudojamos visos Halstead ir McCabe metrikos kartu, bei keletas šių atributų poaibių susiaurintų tiek rankiniu būdų tiek duomenų analizės įrankiais.

\subsection{McCabe sudėtingumo metrikos}
1976-aisiais Thomas J, McCabe išleido tyrimą kuriame pasiūlė naują programinės įrangos kodo sudėtingumo matavimo būdą, pavadintą Ciklomatiniu sudėtingumu (angl. Cyclomatic Complexity). Šis matavimo vienetas kiekybiškai matuoja nepriklausomų kelių per kodo elementus kiekį.

Ciklomatnis sudėtingumas apibrėžiamas, kaip kodo vykdyme egzistuojančių tiesiškai nepriklausomų kelių kiekis. Pavyzdžiui jei kode nėra jokių eiliškumo kontrolės sakinių (angl. Control flow statements), kaip sąlygninai „if“ sakiniai, tuomet kodo ciklomatinis sudėtingumas būtų 1. Su vienu „if“ sąlyginiu sakiniu, po kurio įvykdymo galimi du keliai: jei grąžinta „TRUE“ bei jei grąžinta „FALSE“, kodo ciklomatinis sudėtingumas būtų 2. Du vieną sąlygą turintys „if“ sakiniai ar vienas toks sakinys su dviem salygomis sudarytų grafą kurio ciklomatinis sudėtingmas lygus 3. \cite{McCabe:1976:CM:800253.807712}

Ciklomatinis sudėtingumas skaičiuojamas naudojant orientuotą grafą kuris vaizduoją programos kodą. Tokiu atveju grafo viršūnės atspindi komandų kurios vykdomos kartu grupes, o kraštinė su kryptimi sujungia dvi viršūnes jei jos gali būti vykdomos viena po kitos. Tokį ciklomatinio sudėtingumo skaičiavimo metodą galima taikyti ir aukšstesnės abstrakcijos progaminio kodo vienetams kaip funkcijos, metodai, klasės ar kt.

Matematiškai ciklomatinis sudėtingumas M apskaičiuojamas formule M = E - N + 2P. Kur E = grafo kraštinių kiekis, N = grafo viršūnių kiekis, o P = apjungtų komponentų kiekis. Dar cikomatinis sudėtingumas skaičiuojamas ir formule M = E - N + P. Šiuo atveju kodas vaizduojamas grafu kurio pradinio ir paskutinio modulio grafo viršūnės yra sujungtos, taip sudarant stipriai apjungtą grafą, ir kodo ciklomatinis sudėtingumas prilygsta grafo ciklomatiniam skaičiui.

Pavyzdžiui, turint programą \cite{img:Cyclomatic_graph_1} kurios veikimas prasideda nuo modulio atvaizduoto raudona viršūne, o kodo vykdymas pabaigiamas įvykdžius mėlyna viršūne atvaizduotą kodą. Toks grafas sudarytas iš devynių kraštinių bei aštuonių viršūnių, kurie sudaro vieną apjungtąjį komponentą, tai tokios programos ciklomatinis sudėtingumas būtų 9 - 8 + 2 * 1 = 3.

\begin{figure}[H]
    \centering
    \includegraphics[scale=0.5]{img/Cyclomatic_graph_1}
    \caption{Programinio kodo atvaizdavimas grafu}
    \label{img:Cyclomatic_graph_1}
\end{figure}



\subsection{Halstead sudėtingumo metrikos}

\section{Sudėtingumo metrikų panaudojimas aprašant programinės įrangos defektyvumą nuspėjančią besimokančią mašiną}

\subsection{Sprendimų medžio metodu paremta besimokanti mašina}

\subsection{Giliuoju neuroniniu tinklu paremta besimokanti mašina}

\section{Duomenų atrinkimas}

\subsection{Rankiniu būdu ieškant atributų kurie yra tiesiogiai proporcingi kitiems atributams}

\subsection{Duomenų analizė naudojant įrankį Weka}

\section{Neuroninio tinklo efektyvumo gerinimas naudojant atrinktus duomenis}

\sectionnonum{Rezultatai ir išvados}
Rezultatų ir išvadų dalyje išdėstomi pagrindiniai darbo rezultatai (kažkas
išanalizuota, kažkas sukurta, kažkas įdiegta), toliau pateikiamos išvados
(daromi nagrinėtų problemų sprendimo metodų palyginimai, siūlomos
rekomendacijos, akcentuojamos naujovės). Rezultatai ir išvados pateikiami
sunumeruotų (gali būti hierarchiniai) sąrašų pavidalu. Darbo rezultatai turi
atitikti darbo tikslą.

\printbibliography[heading=bibintoc]  % Šaltinių sąraše nurodoma panaudota
% literatūra, kitokie šaltiniai. Abėcėlės tvarka išdėstomi darbe panaudotų
% (cituotų, perfrazuotų ar bent paminėtų) mokslo leidinių, kitokių publikacijų
% bibliografiniai aprašai. Šaltinių sąrašas spausdinamas iš naujo puslapio.
% Aprašai pateikiami netransliteruoti. Šaltinių sąraše negali būti tokių
% šaltinių, kurie nebuvo paminėti tekste. Šaltinių sąraše rekomenduojame
% necituoti savo kursinio darbo, nes tai nėra oficialus literatūros šaltinis.
% Jei tokių nuorodų reikia, pateikti jas tekste.

% \sectionnonum{Sąvokų apibrėžimai}
\sectionnonum{Santrumpos}
Sąvokų apibrėžimai ir santrumpų sąrašas sudaromas tada, kai darbo tekste
vartojami specialūs paaiškinimo reikalaujantys terminai ir rečiau sutinkamos
santrumpos.

\appendix  % Priedai
% Prieduose gali būti pateikiama pagalbinė, ypač darbo autoriaus savarankiškai
% parengta, medžiaga. Savarankiški priedai gali būti pateikiami ir
% kompaktiniame diske. Priedai taip pat numeruojami ir vadinami. Darbo tekstas
% su priedais susiejamas nuorodomis.

\section{Niauroninio tinklo struktūra}
\begin{figure}[H]
    \centering
    \includegraphics[scale=0.5]{img/MLP}
    \caption{Paveikslėlio pavyzdys}
    \label{img:mlp}
\end{figure}


\section{Eksperimentinio palyginimo rezultatai}
% tablesgenerator.com - converts calculators (e.g. excel) tables to LaTeX
\begin{table}[H]\footnotesize
  \centering
  \caption{Lentelės pavyzdys}
  {\begin{tabular}{|l|c|c|} \hline
    Algoritmas & $\bar{x}$ & $\sigma^{2}$ \\
    \hline
    Algoritmas A  & 1.6335    & 0.5584       \\
    Algoritmas B  & 1.7395    & 0.5647       \\
    \hline
  \end{tabular}}
  \label{tab:table example}
\end{table}

\end{document}
