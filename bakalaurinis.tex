\documentclass{VUMIFPSbakalaurinis}
\usepackage{algorithmicx}
\usepackage{algorithm}
\usepackage{algpseudocode}
\usepackage{amsfonts}
\usepackage{amsmath}
\usepackage{bm}
\usepackage{caption}
\usepackage{color}
\usepackage{float}
\usepackage{graphicx}
\usepackage{listings}
\usepackage{subfig}
\usepackage{wrapfig}

% Titulinio aprašas
\university{Vilniaus universitetas}
\faculty{Matematikos ir informatikos fakultetas}
\department{Programų sistemų katedra}
\papertype{Bakalauro darbas}
\title{Neuroninių tinklų panaudojimas testavimo proceso gerinimui}
\titleineng{Using neural networks to improve the testing process}
\author{Ričardas Mikelionis}
% \secondauthor{Vardonis Pavardonis}   % Pridėti antrą autorių
\supervisor{asist. dr. Vytautas Valaitis}
\reviewer{partn. prof., dr. Aldas Glemža}
\date{Vilnius – \the\year}

% Nustatymai
\setmainfont{Palemonas}   % Pakeisti teksto šriftą į Palemonas (turi būti įdiegtas sistemoje)
\bibliography{bibliografija}

\begin{document}
\maketitle

%% Padėkų skyrius
% \sectionnonumnocontent{}
% \vspace{7cm}
% \begin{center}
%     Padėkos asmenims ir/ar organizacijoms
% \end{center}

\sectionnonumnocontent{Santrauka}
Glaustai aprašomas darbo turinys: pristatoma nagrinėta problema ir padarytos
išvados. Santraukos apimtis ne didesnė nei 0,5 puslapio. Santraukų gale
nurodomi darbo raktiniai žodžiai. 
% Nurodomi iki 5 svarbiausių temos raktinių žodžių (terminų).
% Vienas terminas gali susidėti iš kelių žodžių.
\raktiniaizodziai{raktinis žodis 1, raktinis žodis 2, raktinis žodis 3, raktinis žodis 4, raktinis žodis 5}   

\sectionnonumnocontent{Summary}
Santrauka anglų kalba. Santraukos apimtis ne didesnė nei 0,5 puslapio.
\keywords{keyword 1, keyword 2, keyword 3, keyword 4, keyword 5}

\tableofcontents

\sectionnonum{Įvadas}
Smarkiai augant Continuous Delivery principų populiarumui auga poreikis testuoti daugiau per trumpesnį laiko tarpą. Paprastas sprendimas šiai problemai būtų be abejo nuolat besisukantis automatinių testų paketas, tačiau automatiniais testais padengti programinį kodą 100\% praktiškai neįmanoma. 100\% padengimas įmanomas tik pagal kokią nors specifinę metriką, o vaikytis tikrojo 100\% padengtumo testais pagal kiekvieną metriką ar funkcinę sritį prilygsta šviesos geičio vaikymuisi, kuo arčiau esame tikslo tuo daugiau pastangų ir resursų reikia pasistūmėti į priekį. Todėl, žinoma, vis dar išlieka poreikis rankiniam testavimui. Norint išleisti programinės įrangos versiją kuo dažniau ištestuoti visko kiekvieną kartą neįmanoma. Taip šis problemos sprendimas iškelia dar vieną, mažesnę, problemą: kaip efektyviai pasirinkti testavimo sritis kiekvienai naujai programos laidai (angl. release). Alan M. Davis \cite{Davis:1995:PSD:203406} tegia, jog pareto principą galima pritaikyti ir programinės įrangos testavime: čia 80\% kode esančių defektų surandami 20\% viso kodo.

Tobulame pasaulyje iteracijai pasirinktos testavimo sritys ir apims tuos 20\% problematiškojo kodo. Tačiau dažnai net remiantis visa turima istorine informacija testuotojui atsakingam už testavimo sričių parinkimą yra sunku efektyviai atrinkti testavimo sritis, kuriose atliktas darbas turės didžiausią įtaką programinės įrangos kokybei.

Sieniant sėkmingai įgyvendinti darbo tikslą siekiama įgyvendinti šiuos uždavinius:
\begin{enumerate}
\item Išrinkti metrikas darančias įtaką programinės įrangos sudėtingumui
\item Palyginti efektyvumą tarp sprendimų medžio besimokančios mašinos ir neuroninio tinklo naudojant surinktus duomenis
\item Atrinkti duomenis turinčius mažiausią įtaką galutiniam spėjimui
\item Naudojantis mažesniu atributų kiekiu iš naujo apmokyti neuroninį tinklą ir palyginti rezultatus su pilnų duomenų rinkinių apmokyto neuroninio tinklo
\end{enumerate}
 
\section{Programinio kodo sudėtingumo ir defektyvumo sąsajos}

\subsection{McCabe sudėtingumo metrikos}

\subsection{Halstead sudėtingumo metrikos}

\section{Sudėtingumo metrikų panaudojimas aprašant programinės įrangos defektyvumą nuspėjančią besimokančią mašiną}

\subsection{Sprendimų medžio metodu paremta besimokanti mašina}

\subsection{Giliuoju neuroniniu tinklu paremta besimokanti mašina}

\section{Duomenų atrinkimas}

\subsection{Rankiniu būdu ieškant atributų kurie yra tiesiogiai proporcingi kitiems atributams}

\subsection{Duomenų analizė naudojant įrankį Weka}

\section{Neuroninio tinklo efektyvumo gerinimas naudojant atrinktus duomenis}

\sectionnonum{Rezultatai ir išvados}
Rezultatų ir išvadų dalyje išdėstomi pagrindiniai darbo rezultatai (kažkas
išanalizuota, kažkas sukurta, kažkas įdiegta), toliau pateikiamos išvados
(daromi nagrinėtų problemų sprendimo metodų palyginimai, siūlomos
rekomendacijos, akcentuojamos naujovės). Rezultatai ir išvados pateikiami
sunumeruotų (gali būti hierarchiniai) sąrašų pavidalu. Darbo rezultatai turi
atitikti darbo tikslą.

\printbibliography[heading=bibintoc]  % Šaltinių sąraše nurodoma panaudota
% literatūra, kitokie šaltiniai. Abėcėlės tvarka išdėstomi darbe panaudotų
% (cituotų, perfrazuotų ar bent paminėtų) mokslo leidinių, kitokių publikacijų
% bibliografiniai aprašai. Šaltinių sąrašas spausdinamas iš naujo puslapio.
% Aprašai pateikiami netransliteruoti. Šaltinių sąraše negali būti tokių
% šaltinių, kurie nebuvo paminėti tekste. Šaltinių sąraše rekomenduojame
% necituoti savo kursinio darbo, nes tai nėra oficialus literatūros šaltinis.
% Jei tokių nuorodų reikia, pateikti jas tekste.

% \sectionnonum{Sąvokų apibrėžimai}
\sectionnonum{Santrumpos}
Sąvokų apibrėžimai ir santrumpų sąrašas sudaromas tada, kai darbo tekste
vartojami specialūs paaiškinimo reikalaujantys terminai ir rečiau sutinkamos
santrumpos.

\appendix  % Priedai
% Prieduose gali būti pateikiama pagalbinė, ypač darbo autoriaus savarankiškai
% parengta, medžiaga. Savarankiški priedai gali būti pateikiami ir
% kompaktiniame diske. Priedai taip pat numeruojami ir vadinami. Darbo tekstas
% su priedais susiejamas nuorodomis.

\section{Niauroninio tinklo struktūra}
\begin{figure}[H]
    \centering
    \includegraphics[scale=0.5]{img/MLP}
    \caption{Paveikslėlio pavyzdys}
    \label{img:mlp}
\end{figure}


\section{Eksperimentinio palyginimo rezultatai}
% tablesgenerator.com - converts calculators (e.g. excel) tables to LaTeX
\begin{table}[H]\footnotesize
  \centering
  \caption{Lentelės pavyzdys}
  {\begin{tabular}{|l|c|c|} \hline
    Algoritmas & $\bar{x}$ & $\sigma^{2}$ \\
    \hline
    Algoritmas A  & 1.6335    & 0.5584       \\
    Algoritmas B  & 1.7395    & 0.5647       \\
    \hline
  \end{tabular}}
  \label{tab:table example}
\end{table}

\end{document}
